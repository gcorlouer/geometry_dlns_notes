\documentclass{article}
\usepackage{amsmath}
\usepackage{amssymb}
\usepackage{amsthm}

% Operators
\newcommand{\diag}{\operatorname{diag}}
\newcommand{\Hom}{\operatorname{Hom}}
\newcommand{\Rep}{\operatorname{Rep}}
\newcommand{\rank}{\operatorname{rank}}
\newcommand{\Supp}{\operatorname{Supp}}
\usepackage{graphicx}
\usepackage{geometry}
\usepackage{graphicx}
\usepackage{hyperref}
% Required packages for algorithms
\usepackage{algorithm}
\usepackage{algpseudocode}
\usepackage{subcaption}
\usepackage[normalem]{ulem}
\usepackage{comment}
\usepackage{tikz}
\usetikzlibrary{arrows.meta,positioning}
% theorem/proposition environments
\theoremstyle{definition}
\newtheorem{theorem}{Theorem}[section]
\newtheorem{proposition}[theorem]{Proposition}
\newtheorem{lemma}[theorem]{Lemma}
\newtheorem{corollary}[theorem]{Corollary}
\newtheorem{definition}[theorem]{Definition}
\newtheorem{remark}[theorem]{Remark}
\title{DLN experiments}
\begin{document}

\maketitle

\section{Golden Path hypothesis}
We understand the learning dynamics of gradient flow under assumptions:
\begin{itemize}
    \item Balanced weights (small initialization)
    \item Decoupled modes
    \item Covariance matrices are simultaneously diagonalizable (properties of the data)
\end{itemize}
Under such assumptions, gradient flow is analytically solvable and is described in detail in the supplementary materials of \href{https://arxiv.org/abs/1810.10531}{A mathematical theory of semantic development in deep neural networks
}. With Avi, Strang, etc we extended the learning dynamics to stochastic gradient flow under the same assumption (pdf attached in the overleaf project: Diffusion of SGD on Deep Linear Networks Preprint.pdf). The relevant propositions are proposition 2.5 (modewise SDEs) and proposition 4.2 (induced marginal distribution from Fokker Planck). I would prioritize speed for these experiments at first: qualitative results, descriptive stats before doing proper hypothesis testing.
Some quick and dirty experiment to try first:
\begin{itemize}
    \item Compute difference between rank pattern of a few SGD trajectories and GD
    \item Compute distance as frobenius norm between GD and SGD weight matrices of each layer
    \item Check distance in function space for GD and SGD is small
\end{itemize}
\subsection{Experimental Protocol: Tracking Rank Patterns under GD and SGD.}
There are a lot of internal symmetries, what we should expect instead is that the parameters would have same complexity: weight matrices have same rank patterns. 
Consider a deep linear network of depth $L$ with weights $\{W_\ell(t)\}_{\ell=1}^L$ trained on the same dataset and initialization for both full-batch gradient descent (GD) and stochastic gradient descent (SGD). Fix a global singular value threshold parameter $\varepsilon>0$ (e.g.\ $10^{-5}$) that will be used consistently across all runs and checkpoints.

\medskip
\noindent
\textbf{1. Training and Checkpointing.}
\begin{enumerate}
    \item Run one GD trajectory and $S$ independent SGD trajectories (same initialization, different batch orderings we care about the effect of gradient noise).
    \item Choose a checkpoint alignment rule (recommended: loss-aligned checkpoints at fixed loss values and escape checkpoints defined by the first significant drop from a saddle plateau).
    \item At each selected checkpoint time $t_k$, save all layer weights $\{W_\ell(t_k)\}_{\ell=1}^L$.
\end{enumerate}

\medskip
\noindent
\textbf{2. Rank Pattern Computation.}
For each saved checkpoint:
\begin{enumerate}
    \item For all $1 \le i \le j \le L$, compute the contiguous partial products
    \[
        P_{ij}(t_k) = W_j(t_k)\cdots W_i(t_k).
    \]
    \item Compute the singular values $\{\sigma_m(P_{ij})\}$ and define the numerical rank (or use a ready made python function to compute rank)
    \[
        r_{ij}(t_k) = \#\{m : \sigma_m(P_{ij}) > \varepsilon\,\sigma_1(P_{ij})\}.
    \]
    \item Record the singular value gap at the threshold (ratio $\sigma_r/\sigma_{r+1}$) to monitor rank ambiguity.
\end{enumerate}

\medskip
\noindent
\textbf{3. Multiplicity Pattern (Optional not a priority).}
Compute the multiplicity pattern via the discrete inclusion--exclusion transform
\[
    m_{ij}(t_k) = r_{ij}(t_k) - r_{i,j+1}(t_k) - r_{i-1,j}(t_k) + r_{i-1,j+1}(t_k),
\]
with zero boundary conventions. If small negative values occur due to numerical noise, clip to zero.

\medskip
\noindent
\textbf{4. Aggregation for SGD.}
At each checkpoint $t_k$, compute:
\begin{itemize}
    \item The mean rank pattern $\bar r_{ij}^{\mathrm{SGD}}(t_k)$ across the $S$ runs.
    \item The mean multiplicity pattern $\bar m_{ij}^{\mathrm{SGD}}(t_k)$.
    \item Entrywise standard deviations and bootstrap confidence intervals.
\end{itemize}

\medskip
\noindent
\textbf{5. GD vs.\ SGD Comparison Metrics.}
For each checkpoint compute:
\[
    d_r(t_k) = \sum_{i\le j} \big| r^{\mathrm{GD}}_{ij}(t_k) - \bar r^{\mathrm{SGD}}_{ij}(t_k) \big|,
\]
\[
    d_m(t_k) = \sum_{i\le j} \big| m^{\mathrm{GD}}_{ij}(t_k) - \bar m^{\mathrm{SGD}}_{ij}(t_k) \big|. \quad \text{OPTIONAL}
\]
Plot $d_r(t_k)$ and $d_m(t_k)$ as functions of aligned training time, with confidence bands for SGD. Emphasize behavior in temporal windows surrounding saddle escape events.

\medskip
\noindent
\textbf{Deliverables.}
\begin{itemize}
    \item For different regimes (saddle to saddle, intermediate, NTK) and different data rank: consider teacher matrix that is not full rank, full rank and degenerate.
    \item Heatmaps of $r_{ij}$ for GD and mean SGD at selected checkpoints.
    \item Trajectories of $d_r(t)$ with uncertainty bands.
    \item Diagnostics of singular value gaps to identify unstable rank transitions.
\end{itemize}

\subsection{Experimental Protocol: Singular-Value Patterns (Top--$K$) under GD vs.\ SGD.}
We can also check the singular values of the individual layers for GD and SGD and check that they are approximately, or eventually become approximately the same throughout training.
We compare GD and SGD by tracking the singular values of all contiguous partial products in a deep linear network of depth $L$. Fix a target truncation level $K$ (e.g.\ $K=5$ or $10$) and a numerical floor $\delta>0$ (e.g.\ $\delta=10^{-30}$) used to avoid taking $\log(0)$. Use the \emph{same} initialization $W_\ell(0)$ for GD and for every SGD run; vary only the mini-batch shuffling/permutation seed across SGD runs.

\medskip
\noindent\textbf{1. Training and Checkpointing.}
\begin{enumerate}
    \item Run one full-batch GD trajectory and $S$ independent SGD trajectories (same initialization, different data-order seeds).
    \item Choose a checkpoint alignment rule (recommended: loss-aligned checkpoints at fixed loss values, or escape-aligned checkpoints defined around plateau-to-drop events).
    \item At each checkpoint time $t_k$, save the full set of layer weights $\{W_\ell(t_k)\}_{\ell=1}^L$ for GD and for each SGD run.
\end{enumerate}

\medskip
\noindent\textbf{2. Partial Products and Top--$K$ Spectra.}
For each saved checkpoint $t_k$ (and each run):
\begin{enumerate}
    \item For all $1\le i\le j\le L$, compute the contiguous partial products
    \[
        P_{ij}(t_k) = W_j(t_k)\cdots W_i(t_k).
    \]
    \item Compute the top--$K$ singular values of each $P_{ij}(t_k)$:
    \[
        \sigma^{(ij)}_1(t_k)\ge \sigma^{(ij)}_2(t_k)\ge \cdots \ge \sigma^{(ij)}_K(t_k)\ge 0,
    \]
    using a deterministic SVD routine (or randomized SVD with a fixed tolerance and oversampling parameter).
    \item Form the \emph{log-singular-value pattern} tensor entries
    \[
        s_{ij,k}(t_k)=\log\!\big(\max\{\sigma^{(ij)}_k(t_k),\delta\}\big),\qquad k=1,\dots,K.
    \]
    \item (Optional but recommended) Also record the \emph{shape-normalized} spectrum
    \[
        \tilde s_{ij,k}(t_k)=\log\!\Big(\frac{\max\{\sigma^{(ij)}_k(t_k),\delta\}}{\max\{\sigma^{(ij)}_1(t_k),\delta\}}\Big),
    \]
    to separate scale ($\sigma_1$) from spectral shape.
\end{enumerate}

\medskip
\noindent\textbf{3. Aggregation across SGD Runs.}
At each checkpoint $t_k$, compute the SGD mean patterns entrywise in log-space:
\[
    \bar s^{\mathrm{SGD}}_{ij,k}(t_k)=\frac{1}{S}\sum_{s=1}^S s^{(s)}_{ij,k}(t_k),
    \qquad
    \bar{\tilde s}^{\mathrm{SGD}}_{ij,k}(t_k)=\frac{1}{S}\sum_{s=1}^S \tilde s^{(s)}_{ij,k}(t_k),
\]
and estimate uncertainty using bootstrap confidence intervals over the $S$ runs (resample runs with replacement).

\medskip
\noindent\textbf{4. GD vs.\ SGD Discrepancy Metrics.}
For each checkpoint $t_k$, quantify the discrepancy between GD and SGD using $\ell_1$ distances:
\[
    d_{\sigma}(t_k)=\sum_{1\le i\le j\le L}\sum_{k=1}^K \Big| s^{\mathrm{GD}}_{ij,k}(t_k)-\bar s^{\mathrm{SGD}}_{ij,k}(t_k)\Big|,
\]
and, if using normalized spectra,
\[
    d_{\mathrm{shape}}(t_k)=\sum_{1\le i\le j\le L}\sum_{k=1}^K \Big| \tilde s^{\mathrm{GD}}_{ij,k}(t_k)-\bar{\tilde s}^{\mathrm{SGD}}_{ij,k}(t_k)\Big|.
\]
Optionally use interval-dependent weights $w_{ij}$ (e.g.\ $w_{ij}=j-i+1$) to emphasize long products:
\[
    d_{\sigma,w}(t_k)=\sum_{i\le j} w_{ij}\sum_{k=1}^K \Big| s^{\mathrm{GD}}_{ij,k}(t_k)-\bar s^{\mathrm{SGD}}_{ij,k}(t_k)\Big|.
\]

\medskip
\noindent\textbf{5. Reporting (Focused on Saddle Escape).}
\begin{itemize}
    \item Plot $d_{\sigma}(t_k)$ and (if applicable) $d_{\mathrm{shape}}(t_k)$ versus aligned time, with bootstrap confidence bands from SGD.
    \item Provide heatmaps of $s_{ij,1}(t_k)=\log\sigma^{(ij)}_1(t_k)$ and selected $k>1$ slices (and/or $\tilde s$ slices) at key checkpoints before, during, and after escape.
    \item Report the same analyses in an escape-centered window $t_k\in[t_{\mathrm{esc}}-\Delta,t_{\mathrm{esc}}+\Delta]$ where $t_{\mathrm{esc}}$ is the escape time defined by the chosen alignment rule.
\end{itemize}
\paragraph{Distance in parameter space}
Simply check distance in parameter space, that is, test the difference in parameter space between:
\begin{equation*}
   \mathbf{E}_{B} ||\theta_{GD} - \theta_{SGD}||_2
\end{equation*}
Average over batch partitions. I would also plot $||\theta_{GD}||$ with small initialization to have a better sense of scale (it should correspond to parameters at various saddle during training) as well as the ratio:
\begin{equation*}
   \frac{\mathbf{E}_{B} ||\theta_{GD} - \theta_{SGD}||_2}{||\theta_{GD}||_2}
\end{equation*}
Check cosine similarity (averaged over batching): 
\begin{equation*}
\cos(\theta_{GD},\theta_{SGD})=\frac{\theta_{GD}\cdot\theta_{SGD}}{\lVert\theta_{GD}\rVert_2\,\lVert\theta_{SGD}\rVert_2}
\end{equation*}
Check layerwise distance (might be useful but we might drop it):
\begin{equation*}
    d_l =  \frac{\mathbf{E}_{B} ||W_{l,GD} - W_{l,SGD}||_F}{||W_{l,GD}||_F}
\end{equation*}
and aggregate:
\begin{equation*}
    d =  \frac{1}{L}\sum_l d_l
\end{equation*}
First we want some qualitative results (descriptive: no hypothesis testing). 
Prediction: I expect this to be false at many training times because just because of the internal symmetries: sgd and gd could select different symmetries. However because of mode alignment it might becomes partially true as we cross saddle points until it eventually become true at the end of training. But the most interesting version is that training select model of similar complexity (internal rank patterns and singular values of internal layers).
\paragraph{Distance in function space}
In function space, GPH is true at convergence because the loss is quadratic and convex in function space and we know from \href{https://proceedings.neurips.cc/paper_files/paper/2022/hash/e9d89428e0ef0a70913845b3ae812ee0-Abstract-Conference.html}{Implicit Regularization or Implicit Conditioning? Exact Risk Trajectories of SGD in High Dimensions
} that GPH holds in function space.
We might still want to check 
\begin{equation*}
   \frac{\mathbf{E}_{B} ||W_{GD} - W_{SGD}||_F}{||W_{GD}||_F}
\end{equation*}
Just to be sure.
\paragraph{Balanced assumptions}
To directly assess the balance condition between adjacent layers, we measure the normalized residual of the balance matrix
\[
G_l = W_l W_l^\top - W_{l+1}^\top W_{l+1}.
\]
Specifically, we define
\[
r_l \;=\; \frac{\left\| W_l W_l^\top - W_{l+1}^\top W_{l+1} \right\|_F}
{\left\| W_l W_l^\top \right\|_F + \left\| W_{l+1}^\top W_{l+1} \right\|_F},
 \]
which is invariant to overall weight scaling. Exact balance corresponds to \(r_l = 0\), while small values of \(r_l\) indicate approximate balance up to numerical tolerance. We track \(r_l\) throughout training to evaluate whether adjacent layers satisfy the balance condition.

\paragraph{Mode decoupling}
To empirically test mode coupling (and its eventual suppression), fix a teacher matrix $\mathbf{M}=\mathbf{U}\mathbf{S}\mathbf{V}^\top$ with singular vectors $\{\mathbf{u}_\alpha\}_{\alpha=1}^r$ and $\{\mathbf{v}_\beta\}_{\beta=1}^r$, and let $\mathbf{W}(t)$ denote the student weight matrix at training time $t$. We define the \emph{mode coefficients} by
\[
w_{\alpha\beta}(t)\;\triangleq\;\mathbf{u}_\alpha^\top \mathbf{W}(t)\,\mathbf{v}_\beta,
\]
and call the \emph{cross modes} those with $\alpha\neq \beta$. A convenient scalar summary of cross-coupling is the Frobenius energy off the diagonal,
\[
E_{\mathrm{cross}}(t)\;\triangleq\;\sum_{\alpha\neq \beta}\bigl|w_{\alpha\beta}(t)\bigr|^2
\quad\text{(optionally normalized by }E_{\mathrm{tot}}(t)=\sum_{\alpha,\beta}|w_{\alpha\beta}(t)|^2\text{)}.
\]
In experiments, we record $\{w_{\alpha\beta}(t)\}$ throughout training and verify that (i) for each fixed $\alpha$, the set $\{|w_{\alpha\beta}(t)|:\beta\neq\alpha\}$ decays toward $0$ and (ii) $E_{\mathrm{cross}}(t)$ (or $E_{\mathrm{cross}}(t)/E_{\mathrm{tot}}(t)$) decreases toward $0$ \emph{before} the corresponding diagonal mode $w_{\alpha\alpha}(t)$ becomes large. Concretely, define a small threshold $\varepsilon>0$ and compare the first times
\[
t_{\mathrm{cross}}(\alpha)\;\triangleq\;\inf\Bigl\{t:\max_{\beta\neq\alpha}|w_{\alpha\beta}(t)|\le \varepsilon\Bigr\},
\qquad
t_{\mathrm{diag}}(\alpha)\;\triangleq\;\inf\Bigl\{t:|w_{\alpha\alpha}(t)|\ge c\,s_\alpha\Bigr\},
\]
for a constant $c\in(0,1)$ (e.g.\ $c=0.5$) and teacher singular value $s_\alpha$; the empirical claim is that typically $t_{\mathrm{cross}}(\alpha)<t_{\mathrm{diag}}(\alpha)$ across modes $\alpha$, seeds, and batch partitions. We report trajectories of $\max_{\beta\neq\alpha}|w_{\alpha\beta}(t)|$, $|w_{\alpha\alpha}(t)|$, and $E_{\mathrm{cross}}(t)$ (mean $\pm$ s.e.m.\ over runs), thereby directly checking that all cross modes are driven to zero prior to (or at least no later than) the growth of the corresponding aligned diagonal modes.


\section{Sampling saddle points}

We leverage the algebraic parameterization of first-order critical points in deep linear networks (DLNs) given by the rank condition and the cyclic conditions. The objective is to turn these polynomial constraints into a numerical sampler that produces many (typically gauge-inequivalent) saddle points.

\paragraph{Rank strata and target projection.}
Let the teacher (or population) matrix be
\[
M = U \,\mathrm{diag}(s_1,\dots,s_{d_{\mathrm{out}}})\, V^\top,
\]
and fix a rank \(r\) and a subset \(S\) of singular directions with \(|S|=r\). Define \(U_S\) as the columns of \(U\) indexed by \(S\), and the corresponding projector \(P_S := U_S U_S^\top\). A rank-\(r\) critical stratum is characterized by the end-to-end product
\[
W := W_L\cdots W_1 = P_S M, 
\qquad 
\mathrm{rank}(W)=r.
\]
In practice, we choose \(r<r_{\max}\) to avoid global minimizers and bias toward saddle points.

\paragraph{Gauge-fixed normal form and residual variables.}
We work in a fixed representative of the hidden-layer gauge orbit. Each hidden dimension \(d_\ell\) is split into a signal part of dimension \(r\) and a residual part of dimension \(d_\ell-r\). In this gauge, the layers are parameterized as
\[
W_\ell \;\sim\;
\begin{pmatrix}
I_r & 0\\
0 & Z_\ell
\end{pmatrix},
\qquad \ell=2,\dots,L-1,
\]
with residual blocks \(Z_\ell\in\mathbb{R}^{(d_\ell-r)\times(d_{\ell-1}-r)}\). The first and last layers embed the signal path and attach the residuals:
\[
W_L = [\, U_S \;\; U_Q Z_L \,],\qquad
W_1 = 
\begin{pmatrix}
U_S^\top M\\
Z_1
\end{pmatrix},
\]
where \(U_Q\) spans the orthogonal complement of \(U_S\). The signal blocks are chosen so that the product along the signal path equals \(P_S M\) exactly, ensuring the rank condition by construction.

\paragraph{Cyclic and fiber constraints on the residuals.}
In this normal form, first-order criticality imposes polynomial constraints on the residual blocks \(Z=(Z_1,\dots,Z_L)\):

\begin{itemize}
\item \textbf{Fiber (product) constraint:}
\[
\mu(Z) := Z_L Z_{L-1}\cdots Z_1 = 0.
\]

\item \textbf{Cyclic constraints:}
\[
d\mu_\ell(Z) := Z_{\ell-1}\cdots Z_1 \, \sigma \, Z_L\cdots Z_{\ell+1} = 0,
\qquad \ell=1,\dots,L,
\]
where \(\sigma\) is the fixed matrix appearing in the theoretical characterization (e.g. \(\sigma=\Sigma_{XY}U_Q\) in the population setting, or an empirical analogue).
\end{itemize}

\paragraph{Energy formulation.}
To convert these matrix equalities into a scalar objective, we introduce the constraint energy
\[
E(Z)
\;=\;
\frac{1}{2}\|\mu(Z)\|_F^2
\;+\;
\frac{1}{2}\sum_{\ell=1}^L \|d\mu_\ell(Z)\|_F^2.
\]
Then \(E(Z)=0\) if and only if all cyclic and fiber constraints are satisfied exactly.

\paragraph{Ensuring saddle points.}
A cheap sufficient construction for producing saddle points is to enforce two distinct residual blocks to be zero. Concretely, choose \(a\neq b\) and set
\[
Z_a=0,\qquad Z_b=0,
\]
while sampling the remaining \(Z_\ell\). This automatically enforces the fiber constraint and typically satisfies all cyclic constraints. 

\paragraph{Boltzmann sampling over constrained residuals.}
To sample many solutions rather than a single feasible point, we sample from a Boltzmann distribution concentrating near the constraint variety:
\[
p_\beta(Z) \;\propto\; \exp\big(-\beta\,E(Z)\big)\,\pi(Z),
\]
where \(\beta>0\) controls concentration and \(\pi(Z)\) is a prior that prevents collapse to trivial solutions. Typical choices include Gaussian or Gaussian with weight decay
\[
\log \pi(Z) = -\frac{\alpha}{2}\sum_{\ell=1}^L \|Z_\ell\|_F^2,
\]
or norm-targeting penalties that keep \(\|Z_\ell\|_F\) away from zero.

\paragraph{SGLD updates.}
Define the negative log-density (up to an additive constant)
\[
\mathcal{L}_\beta(Z) := \beta\,E(Z) - \log \pi(Z).
\]
Stochastic gradient Langevin dynamics (SGLD) then takes the form
\[
Z_{t+1}
=
Z_t
-
\frac{\epsilon_t}{2}\nabla \mathcal{L}_\beta(Z_t)
+
\eta_t,
\qquad
\eta_t \sim \mathcal{N}(0,\epsilon_t I),
\]
with stepsize schedule \(\epsilon_t\). When empirical analogues of the constraints are used, gradients can be estimated on minibatches.

\paragraph{Projection to true DLN critical points and Hessian certification.}
\textbf{Comment: I'm not sure about the usefulness of doing this}: Because the parameterization and the matrix \(\sigma\) may be approximate in finite-sample settings, after obtaining a low-energy configuration \(Z\) we assemble the full weights \(W_1,\dots,W_L\) and perform a short refinement step on the true DLN loss to reach \(\|\nabla \mathcal{\mathcal{L}}\|\approx 0\), while periodically re-balancing layer norms to control gauge drift. \textbf{Comment: this is going to be useful:} finally, saddle points are certified by estimating the smallest eigenvalues of the Hessian using Hessian-vector products; the presence of a negative eigenvalue indicates a strict saddle and a non strict saddle otherwise (or perhap§s local minimum in the underparametrize regime).
\end{document}
